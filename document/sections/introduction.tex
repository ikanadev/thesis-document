Vivimos en un mundo digital online, un gran parte de las actividades que realizamos día a día
están fúertemente ligadas a ella, ya sea desde nuestro smartphone o computador, vivimos
conectados a Internet.

Prueba de ello es nuestro día a día, vivimos tan ligados al mundo digital que
esta incurre en cada aspecto de nuestra vida, desde el trabajo, hasta las relaciones
personales, pasando por entretenimiento, hobbies, música, estudios, y un largo etcétera.

Por tal motivo, hoy en día, todo negocio o empresa que quiera darse a conocer se ve 
obligada a tener presencia online, dado que cada vez menos personas interactúan
con medios tradicionales (periódico, radio, televisión), y esta característica se 
hace notar más en las nuevas generaciones.

La opción más económica y accesible para tener presencia online es el uso de redes
sociales, y de hecho casi optan por esta opción al iniciar, pero esta tiene 2 grandes
desventajar, la primera es que carece de fiabilidad dado que, prácticamente cualquier
persona puede crear una página en dichar redes, lo cual, en primera instancia, pone
en duda la veracidad de una página en redes sociales; la segunda es que como dueños
del negocio o empresa, estamos sujetos a las normas y estructura de estas
redes, lo cual limita de gran manera la presentación visual del negocio o empresa.
Una tendencia actual que se maneja, es la de usar redes sociales como principal
plataforma de comunicación y usar el sitio web como una instancia donde el
usuario podrá acceder a información más específica y ver con más detalle 
los productos y/o servicios que posea la empresa. 

Por tal motivo, tomando en cuenta todo lo indicado anteriormente, en el presente
proyecto se desarrollará e implementará un sistema web para adminstrar, gestionar
y dar a conocer al publico en general
el contenido, docentes y cursos dictados en el Laboratorio Superior de Informática
(LASIN).