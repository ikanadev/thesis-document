\section{Institución}
En 1973 bajo el nuevo Sistema de Universidades se crean las menciones de Estadística
e Informática como una menciones del Departamento de Matemática, según el certificado
emitido Por el Comité Ejecutivo de la Universidad Boliviana (CEUB) Nro. 279/2013.

A finales de los años 70 se realiza la compra del ordenador PDP-11/34
(CPU: KA11 hasta KD11, o el LSI-11; RAM: 8Kbytes; Almacenamiento: DECtape,
Magtape, y discos RS64, RS11, RK05 y RP02), a inicios de los años 80 comienza
las intervenciones militares hechas por el entonces presidente de Bolivia Luis
García Meza Tejada, en la cual la Educación Superior sufre un proceso de
transformación política y académica muy profunda, por lo que el ordenador
PDP-11/34 que pertenecía a la mención de Informática fue trasladada al
centro de Cálculo de la Carrera de Ingeniería. Tras la finalización de
los Gobiernos Militares se creó el Concejo Nacional de Reforma Universitaria
(CNRU) antecesor del Consejo Nacional de Educación Superior (CNES), que procedería
a elaborar la Ley Fundamental de Universidades, que modificaría la estructura de
la Universidades Públicas del país, según se menciona en la Reseña Histórica de
la Facultad de Ciencias Puras y Naturales (2012:29). En 1983 la mención de
Informática se independiza administrativamente de la Carrera de Matemáticas,
convirtiéndose posteriormente en Carrera de Informática esto debido al rápido
incremento de los estudiantes interesados en las nuevas tecnologías emergentes
y su rápida extensión en las empresas nacionales, la Carrera de Informática comenzó
con aproximadamente 300 estudiantes. Y por resolución del 6to Congreso de
Universidades se valida la conversión de las menciones de Estadística e
Informática en Carreras independientes, es así que durante este año la
Carrera de Informática empieza a funcionar en el edificio del segundo
patio del Monoblock Central de la UMSA, donde antes funcionaba el departamento
de Administrativo Financiero (DAF), durante este año la Carrera de Informática
hace las respectivas gestiones para que el ordenador PDP-11/34 retorne a la
Carrera y en esos años también se creó el LASIN cuyos acrónimos significan
Laboratorio Superior en Informática, denominado así por el Lic. Suarez como
una instancia superior de la Carrera de Informática; el Laboratorio Superior
en Informática comenzó sus actividades con el ordenador PDP-11/34 con la cual
los estudiantes hacían las prácticas en el Lenguaje Basic, la cual se utilizó
para la transcripción de actas para la Facultad y las demás Carreras, considerándose
así como la primera computadora del LASIN, pero también se fueron adquiriendo más
ordenadores, uno de los cuales fue el ordenador Huamp y el ordenador SEMIK; después
de la consolidación del LASIN el Lic. Eufren Llanque se convierte en el responsable
del LASIN y es nombrado jefe del Laboratorio Superior en Informática, e inicia sus
funciones con un personal de colaboración conformado por el Lic. Efraín Silva (+),
Lic. Flores y la Lic. Virginia Aramayo, los cuales fueron considerados como JTP
(Jefe de Trabajo Práctico) y se empieza a trabajar con un Sistema RTS en una
pequeña sala de la Carrera.

En los años 90 llega la epidemia de compra de computadoras en la Universidad, en
la cual la Carrera de Informática aprovecha la oportunidad de la adquisición de
estos medios de Tecnología, y envía solicitudes para la compra de ordenadores
multiusuario, en las cuales se empezó a trabajar en los lenguajes de Pascal, Cobol,
Fortran, Pl/1, Lisp, Algol, Basic, entre otros. Los estudiantes se quedaban hasta
altas horas de la noche para hacer uso de los ordenadores que se habían adquirido,
esto por el costo elevado de los ordenadores y además era una forma de aprender y
adquirir más conocimiento, años más tarde el LASIN adquiere los ordenadores 286.

Después se toma la iniciativa de adquirir 30 ordenadores, con los cuales se decide
brindar cursos especializados para la enseñanza a docentes, estudiantes,
administrativos y público en general, cada vez aumentaba el número de estudiantes
que querían aprender nuevas tecnologías en el desarrollo y por lo que se va
adquiriendo más computadoras.

El LASIN ha ido brindando curso especializados de Microsoft SQL Server, php MySQL,
Diseño de Páginas Web, HTML básico, HTML5, Java, Excel, Android, y otros cursos a
estudiantes de las distintas Carreras de la Universidad Mayor de San Andrés como
también a personas externas a la Universidad, estos cursos fueron dictados por
módulos durante los semestres y en distintos horarios; el LASIN también es utilizado
para los cursos Pre-Universitarios de computación y las materias de Programación
de la Carrera de Informática.
\section{Proyectos Similares}
Proyectos similares
% https://repositorio.umsa.bo/bitstream/handle/123456789/12560/T.3292.pdf?sequence=1&isAllowed=y
% https://repositorio.umsa.bo/bitstream/handle/123456789/17598/T-3441.pdf?sequence=1&isAllowed=y
% https://repositorio.umsa.bo/bitstream/handle/123456789/17471/T-3403.pdf?sequence=1&isAllowed=y
% https://repositorio.umsa.bo/bitstream/handle/123456789/17530/T-3434.pdf?sequence=1&isAllowed=y
% https://repositorio.umsa.bo/bitstream/handle/123456789/17484/T-3417.pdf?sequence=1&isAllowed=y
